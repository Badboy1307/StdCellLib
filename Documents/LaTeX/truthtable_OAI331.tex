%%  ************    LibreSilicon's StdCellLibrary   *******************
%%
%%  Organisation:   Chipforge
%%                  Germany / European Union
%%
%%  Profile:        Chipforge focus on fine System-on-Chip Cores in
%%                  Verilog HDL Code which are easy understandable and
%%                  adjustable. For further information see
%%                          www.chipforge.org
%%                  there are projects from small cores up to PCBs, too.
%%
%%  File:           StdCellLib/Documents/LaTeX/truthtable_OAI331.tex
%%
%%  Purpose:        Truth Table File for OAI331
%%
%%  ************    LaTeX with circdia.sty package      ***************
%%
%%  ///////////////////////////////////////////////////////////////////
%%
%%  Copyright (c) 2018 by chipforge <hsank@nospam.chipforge.org>
%%  All rights reserved.
%%
%%      This Standard Cell Library is licensed under the Libre Silicon
%%      public license; you can redistribute it and/or modify it under
%%      the terms of the Libre Silicon public license as published by
%%      the Libre Silicon alliance, either version 1 of the License, or
%%      (at your option) any later version.
%%
%%      This design is distributed in the hope that it will be useful,
%%      but WITHOUT ANY WARRANTY; without even the implied warranty of
%%      MERCHANTABILITY or FITNESS FOR A PARTICULAR PURPOSE.
%%      See the Libre Silicon Public License for more details.
%%
%%  ///////////////////////////////////////////////////////////////////

\begin{center}
    {\(Z = \lnot ((C2 \lor C1 \lor C0) \land (B2 \lor B1 \lor B0) \land A) \)}
    \begin{table}[h] %\caption{\(Z = \lnot ((C2 \lor C1 \lor C0) \land (B2 \lor B1 \lor B0) \land A) \)}
        \begin{center}
            \begin{tabular}{|c|c|c|c|c|c|c||c|} \hline
            C2 & C1 & C0 & B2 & B1 & B0 & A & Z \\ \hline\hline
            0  & 0  & 0  & X  & X  & X  & X & 1 \\ \hline
            1  & X  & X  & 1  & X  & X  & 1 & 0 \\ \hline
            1  & X  & X  & X  & 1  & X  & 1 & 0 \\ \hline
            1  & X  & X  & X  & X  & 1  & 1 & 0 \\ \hline
            X  & 1  & X  & 1  & X  & X  & 1 & 0 \\ \hline
            X  & 1  & X  & X  & 1  & X  & 1 & 0 \\ \hline
            X  & 1  & X  & X  & X  & 1  & 1 & 0 \\ \hline
            X  & X  & 1  & 1  & X  & X  & 1 & 0 \\ \hline
            X  & X  & 1  & X  & 1  & X  & 1 & 0 \\ \hline
            X  & X  & 1  & X  & X  & 1  & 1 & 0 \\ \hline
            X  & X  & X  & 0  & 0  & 0  & X & 1 \\ \hline
            X  & X  & X  & X  & X  & X  & 0 & 1 \\ \hline
            \end{tabular}
        \end{center}
    \end{table}
\end{center}
