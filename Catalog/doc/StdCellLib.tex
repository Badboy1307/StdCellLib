%%  ************    LibreSilicon's StdCellLibrary   *******************
%%
%%  Organisation:   Chipforge
%%                  Germany / European Union
%%
%%  Profile:        Chipforge focus on fine System-on-Chip Cores in
%%                  Verilog HDL Code which are easy understandable and
%%                  adjustable. For further information see
%%                          www.chipforge.org
%%                  there are projects from small cores up to PCBs, too.
%%
%%  File:           StdCellLib/Documents/LaTeX/stdcelllib.tex
%%
%%  Purpose:        Top Level File for Standard Cell Library Documentation
%%
%%  ************    LaTeX with circdia.sty package      ***************
%%
%%  ///////////////////////////////////////////////////////////////////
%%
%%  Copyright (c) 2018 by chipforge <hsank@nospam.chipforge.org>
%%  All rights reserved.
%%
%%      This Standard Cell Library is licensed under the Libre Silicon
%%      public license; you can redistribute it and/or modify it under
%%      the terms of the Libre Silicon public license as published by
%%      the Libre Silicon alliance, either version 1 of the License, or
%%      (at your option) any later version.
%%
%%      This design is distributed in the hope that it will be useful,
%%      but WITHOUT ANY WARRANTY; without even the implied warranty of
%%      MERCHANTABILITY or FITNESS FOR A PARTICULAR PURPOSE.
%%      See the Libre Silicon Public License for more details.
%%
%%  ///////////////////////////////////////////////////////////////////
\documentclass[10pt,a4paper,twoside]{article}
\usepackage[utf8]{inputenc}
\usepackage[english]{babel}
%\usepackage{amsmath}
%\usepackage{amsfonts}
\usepackage{amssymb}
%\usepackage{gensymb}
%\usepackage{graphicx}
\usepackage[digital,srcmeas,semicon]{circdia}
% \usepackage[dvipsnames]{xcolor}
\usepackage[left=2cm,right=2cm,top=2cm,bottom=2cm]{geometry}

\title{LibreSilicon Standard Cell Library}
\author{Hagen Sankowski}
\date{\today}

\makeindex  % usefull for ToC
\setlength{\parindent}{0pt} % get rid of annoying indents

\begin{document}
\maketitle
\begin{abstract}
\begin{quote}
Copyright \textcopyright  2018 CHIPFORGE.ORG. All rights reserved.

This process is licensed under the Libre Silicon public license; you can redistribute it and/or modify it under the terms of the Libre Silicon public license as published by the Libre Silicon alliance either version 2 of the License, or (at your option) any later version.

This design is distributed in the hope that it will be useful, but WITHOUT ANY WARRANTY; without even the implied warranty of MERCHANTABILITY or FITNESS FOR A PARTICULAR PURPOSE. See the Libre Silicon Public License for more details.

For further clarification consult the complete documentation of the process.
\end{quote}
\end{abstract}

\clearpage
\tableofcontents
\clearpage

\pagestyle{headings}

\section{Considerations}
\newcommand{\stacktfour}{YES}
%\newcommand{\stacktfour}{NO}
\clearpage

\section{Logical Cells}
\twocolumn

\input{cells.tex}
\clearpage


%\onecolumn
%\section{Physical Cells}

%\twocolumn
%%%  ************    LibreSilicon's StdCellLibrary   *******************
%%
%%  Organisation:   Chipforge
%%                  Germany / European Union
%%
%%  Profile:        Chipforge focus on fine System-on-Chip Cores in
%%                  Verilog HDL Code which are easy understandable and
%%                  adjustable. For further information see
%%                          www.chipforge.org
%%                  there are projects from small cores up to PCBs, too.
%%
%%  File:           StdCellLib/Documents/LaTeX/manpage_TIE0.tex
%%
%%  Purpose:        Manual Page File for TIE0
%%
%%  ************    LaTeX with circdia.sty package      ***************
%%
%%  ///////////////////////////////////////////////////////////////////
%%
%%  Copyright (c) 2018 by chipforge <hsank@nospam.chipforge.org>
%%  All rights reserved.
%%
%%      This Standard Cell Library is licensed under the Libre Silicon
%%      public license; you can redistribute it and/or modify it under
%%      the terms of the Libre Silicon public license as published by
%%      the Libre Silicon alliance, either version 1 of the License, or
%%      (at your option) any later version.
%%
%%      This design is distributed in the hope that it will be useful,
%%      but WITHOUT ANY WARRANTY; without even the implied warranty of
%%      MERCHANTABILITY or FITNESS FOR A PARTICULAR PURPOSE.
%%      See the Libre Silicon Public License for more details.
%%
%%  ///////////////////////////////////////////////////////////////////
\label{TIE0}
\paragraph{Cell}
\begin{quote}
    \textbf{TIE0} - a Tie-low (or pull-down) cell 
\end{quote}

\paragraph{Synopsys}
\begin{quote}
    TIE0(Z)
\end{quote}

\paragraph{Description}
%%  ************    LibreSilicon's StdCellLibrary   *******************
%%
%%  Organisation:   Chipforge
%%                  Germany / European Union
%%
%%  Profile:        Chipforge focus on fine System-on-Chip Cores in
%%                  Verilog HDL Code which are easy understandable and
%%                  adjustable. For further information see
%%                          www.chipforge.org
%%                  there are projects from small cores up to PCBs, too.
%%
%%  File:           StdCellLib/Documents/LaTeX/circuit_TIE0.tex
%%
%%  Purpose:        Circuit File for TIE0
%%
%%  ************    LaTeX with circdia.sty package      ***************
%%
%%  ///////////////////////////////////////////////////////////////////
%%
%%  Copyright (c) 2018 by chipforge <hsank@nospam.chipforge.org>
%%  All rights reserved.
%%
%%      This Standard Cell Library is licensed under the Libre Silicon
%%      public license; you can redistribute it and/or modify it under
%%      the terms of the Libre Silicon public license as published by
%%      the Libre Silicon alliance, either version 1 of the License, or
%%      (at your option) any later version.
%%
%%      This design is distributed in the hope that it will be useful,
%%      but WITHOUT ANY WARRANTY; without even the implied warranty of
%%      MERCHANTABILITY or FITNESS FOR A PARTICULAR PURPOSE.
%%      See the Libre Silicon Public License for more details.
%%
%%  ///////////////////////////////////////////////////////////////////
\begin{center}
    Circuit
    \begin{figure}[h] %\caption{Circuit}
        \begin{center}
            \begin{circuitdiagram}{8}{8}
            \resis{2}{4}{V}{R}{}   % pull down R
            \ground{2}{0.5}{D}
            \wire{2}{7}{6}{7}   % pin Z
            \pin{7}{7}{R}{Z}   % pin Z
            \end{circuitdiagram}
        \end{center}
    \end{figure}
\end{center}

%\input{TIE0_schematic.tex}

\paragraph{Truth Table}
%%  ************    LibreSilicon's StdCellLibrary   *******************
%%
%%  Organisation:   Chipforge
%%                  Germany / European Union
%%
%%  Profile:        Chipforge focus on fine System-on-Chip Cores in
%%                  Verilog HDL Code which are easy understandable and
%%                  adjustable. For further information see
%%                          www.chipforge.org
%%                  there are projects from small cores up to PCBs, too.
%%
%%  File:           StdCellLib/Documents/LaTeX/truthtable_TIE0.tex
%%
%%  Purpose:        Truth Table File for TIE0
%%
%%  ************    LaTeX with circdia.sty package      ***************
%%
%%  ///////////////////////////////////////////////////////////////////
%%
%%  Copyright (c) 2018 by chipforge <hsank@nospam.chipforge.org>
%%  All rights reserved.
%%
%%      This Standard Cell Library is licensed under the Libre Silicon
%%      public license; you can redistribute it and/or modify it under
%%      the terms of the Libre Silicon public license as published by
%%      the Libre Silicon alliance, either version 1 of the License, or
%%      (at your option) any later version.
%%
%%      This design is distributed in the hope that it will be useful,
%%      but WITHOUT ANY WARRANTY; without even the implied warranty of
%%      MERCHANTABILITY or FITNESS FOR A PARTICULAR PURPOSE.
%%      See the Libre Silicon Public License for more details.
%%
%%  ///////////////////////////////////////////////////////////////////
\begin{center}
    {\(Z = 0 \)}
    \begin{table}[h] %\caption{\(Z = 0 \)}
        \begin{center}
            \begin{tabular}{||c|} \hline
            Z \\ \hline\hline
            0 \\ \hline
            \end{tabular}
        \end{center}
    \end{table}
\end{center}


\paragraph{Usage}

\paragraph{Fan-in / Fan-out}

\paragraph{Layout}

\paragraph{Files}

\paragraph{See also}
\begin{quote}
    TIE1 - a Tie-high (or pull-up) cell
\end{quote}

%%%  ************    LibreSilicon's StdCellLibrary   *******************
%%
%%  Organisation:   Chipforge
%%                  Germany / European Union
%%
%%  Profile:        Chipforge focus on fine System-on-Chip Cores in
%%                  Verilog HDL Code which are easy understandable and
%%                  adjustable. For further information see
%%                          www.chipforge.org
%%                  there are projects from small cores up to PCBs, too.
%%
%%  File:           StdCellLib/Documents/LaTeX/manpage_TIE1.tex
%%
%%  Purpose:        Manual Page File for TIE1
%%
%%  ************    LaTeX with circdia.sty package      ***************
%%
%%  ///////////////////////////////////////////////////////////////////
%%
%%  Copyright (c) 2018 by chipforge <hsank@nospam.chipforge.org>
%%  All rights reserved.
%%
%%      This Standard Cell Library is licensed under the Libre Silicon
%%      public license; you can redistribute it and/or modify it under
%%      the terms of the Libre Silicon public license as published by
%%      the Libre Silicon alliance, either version 1 of the License, or
%%      (at your option) any later version.
%%
%%      This design is distributed in the hope that it will be useful,
%%      but WITHOUT ANY WARRANTY; without even the implied warranty of
%%      MERCHANTABILITY or FITNESS FOR A PARTICULAR PURPOSE.
%%      See the Libre Silicon Public License for more details.
%%
%%  ///////////////////////////////////////////////////////////////////
\label{TIE1}
\paragraph{Cell}
\begin{quote}
    \textbf{TIE1} - a Tie-high (or pull-up) cell 
\end{quote}

\paragraph{Synopsys}
\begin{quote}
    TIE1(Z)
\end{quote}

\paragraph{Description}
%%  ************    LibreSilicon's StdCellLibrary   *******************
%%
%%  Organisation:   Chipforge
%%                  Germany / European Union
%%
%%  Profile:        Chipforge focus on fine System-on-Chip Cores in
%%                  Verilog HDL Code which are easy understandable and
%%                  adjustable. For further information see
%%                          www.chipforge.org
%%                  there are projects from small cores up to PCBs, too.
%%
%%  File:           StdCellLib/Documents/LaTeX/circuit_TIE1.tex
%%
%%  Purpose:        Circuit File for TIE1
%%
%%  ************    LaTeX with circdia.sty package      ***************
%%
%%  ///////////////////////////////////////////////////////////////////
%%
%%  Copyright (c) 2018 by chipforge <hsank@nospam.chipforge.org>
%%  All rights reserved.
%%
%%      This Standard Cell Library is licensed under the Libre Silicon
%%      public license; you can redistribute it and/or modify it under
%%      the terms of the Libre Silicon public license as published by
%%      the Libre Silicon alliance, either version 1 of the License, or
%%      (at your option) any later version.
%%
%%      This design is distributed in the hope that it will be useful,
%%      but WITHOUT ANY WARRANTY; without even the implied warranty of
%%      MERCHANTABILITY or FITNESS FOR A PARTICULAR PURPOSE.
%%      See the Libre Silicon Public License for more details.
%%
%%  ///////////////////////////////////////////////////////////////////
\begin{center}
    Circuit
    \begin{figure}[h] %\caption{Circuit}
        \begin{center}
            \begin{circuitdiagram}{8}{8}
            \resis{2}{4}{V}{R}{}   % pull up R
            \power{2}{7.5}{U}{}
            \wire{2}{1}{6}{1}   % pin Z
            \pin{7}{1}{R}{Z}   % pin Z
            \end{circuitdiagram}
        \end{center}
    \end{figure}
\end{center}

%\input{TIE1_schematic.tex}

\paragraph{Truth Table}
%%  ************    LibreSilicon's StdCellLibrary   *******************
%%
%%  Organisation:   Chipforge
%%                  Germany / European Union
%%
%%  Profile:        Chipforge focus on fine System-on-Chip Cores in
%%                  Verilog HDL Code which are easy understandable and
%%                  adjustable. For further information see
%%                          www.chipforge.org
%%                  there are projects from small cores up to PCBs, too.
%%
%%  File:           StdCellLib/Documents/LaTeX/truthtable_TIE1.tex
%%
%%  Purpose:        Truth Table File for TIE1
%%
%%  ************    LaTeX with circdia.sty package      ***************
%%
%%  ///////////////////////////////////////////////////////////////////
%%
%%  Copyright (c) 2018 by chipforge <hsank@nospam.chipforge.org>
%%  All rights reserved.
%%
%%      This Standard Cell Library is licensed under the Libre Silicon
%%      public license; you can redistribute it and/or modify it under
%%      the terms of the Libre Silicon public license as published by
%%      the Libre Silicon alliance, either version 1 of the License, or
%%      (at your option) any later version.
%%
%%      This design is distributed in the hope that it will be useful,
%%      but WITHOUT ANY WARRANTY; without even the implied warranty of
%%      MERCHANTABILITY or FITNESS FOR A PARTICULAR PURPOSE.
%%      See the Libre Silicon Public License for more details.
%%
%%  ///////////////////////////////////////////////////////////////////
\begin{center}
    {\(Z = 1 \)}
    \begin{table}[h] %\caption{\(Z = 1 \)}
        \begin{center}
            \begin{tabular}{||c|} \hline
            Z \\ \hline\hline
            1 \\ \hline
            \end{tabular}
        \end{center}
    \end{table}
\end{center}


\paragraph{Usage}

\paragraph{Fan-in / Fan-out}

\paragraph{Layout}

\paragraph{Files}

\paragraph{See also}
\begin{quote}
    TIE0 - a Tie-low (or pull-down) cell
\end{quote}

%%%  ************    LibreSilicon's StdCellLibrary   *******************
%%
%%  Organisation:   Chipforge
%%                  Germany / European Union
%%
%%  Profile:        Chipforge focus on fine System-on-Chip Cores in
%%                  Verilog HDL Code which are easy understandable and
%%                  adjustable. For further information see
%%                          www.chipforge.org
%%                  there are projects from small cores up to PCBs, too.
%%
%%  File:           StdCellLib/Documents/LaTeX/manpage_FILL.tex
%%
%%  Purpose:        Manual Page File for FILL
%%
%%  ************    LaTeX with circdia.sty package      ***************
%%
%%  ///////////////////////////////////////////////////////////////////
%%
%%  Copyright (c) 2018 by chipforge <hsank@nospam.chipforge.org>
%%  All rights reserved.
%%
%%      This Standard Cell Library is licensed under the Libre Silicon
%%      public license; you can redistribute it and/or modify it under
%%      the terms of the Libre Silicon public license as published by
%%      the Libre Silicon alliance, either version 1 of the License, or
%%      (at your option) any later version.
%%
%%      This design is distributed in the hope that it will be useful,
%%      but WITHOUT ANY WARRANTY; without even the implied warranty of
%%      MERCHANTABILITY or FITNESS FOR A PARTICULAR PURPOSE.
%%      See the Libre Silicon Public License for more details.
%%
%%  ///////////////////////////////////////////////////////////////////
\label{FILL}
\paragraph{Cell}
\begin{quote}
    \textbf{FILL} - a Filler cell with capacitance
\end{quote}

\paragraph{Synopsys}
\begin{quote}
    FILL
\end{quote}

\paragraph{Description}
%%  ************    LibreSilicon's StdCellLibrary   *******************
%%
%%  Organisation:   Chipforge
%%                  Germany / European Union
%%
%%  Profile:        Chipforge focus on fine System-on-Chip Cores in
%%                  Verilog HDL Code which are easy understandable and
%%                  adjustable. For further information see
%%                          www.chipforge.org
%%                  there are projects from small cores up to PCBs, too.
%%
%%  File:           StdCellLib/Documents/LaTeX/schematic_FILL.tex
%%
%%  Purpose:        Schematic File for FILL
%%
%%  ************    LaTeX with circdia.sty package      ***************
%%
%%  ///////////////////////////////////////////////////////////////////
%%
%%  Copyright (c) 2018 by chipforge <hsank@nospam.chipforge.org>
%%  All rights reserved.
%%
%%      This Standard Cell Library is licensed under the Libre Silicon
%%      public license; you can redistribute it and/or modify it under
%%      the terms of the Libre Silicon public license as published by
%%      the Libre Silicon alliance, either version 1 of the License, or
%%      (at your option) any later version.
%%
%%      This design is distributed in the hope that it will be useful,
%%      but WITHOUT ANY WARRANTY; without even the implied warranty of
%%      MERCHANTABILITY or FITNESS FOR A PARTICULAR PURPOSE.
%%      See the Libre Silicon Public License for more details.
%%
%%  ///////////////////////////////////////////////////////////////////
\begin{center}
    Schematic (one stage, 2T total)
    \begin{figure}[h] %\caption{Schematic}
        \begin{center}
            \begin{circuitdiagram}{10}{15}
            \trans{nenh*}{6}{4}{R}{$M_{N}$}{}
            \trans{penh*}{5}{10}{L}{}{$M_{P}$}
            \ground{8}{0.5}{D}  % ground below nmos
            \power{3}{13.5}{U}{}  % power above pmos
            \wire{8}{6}{8}{11.5}  % wire between pmos gate and nmos
            \wire{3}{2.5}{3}{8}   % wire between nmos gate and pmos
            \end{circuitdiagram}
        \end{center}
    \end{figure}
\end{center}


\paragraph{Truth Table}
\begin{quote}
No Truth Table applicable.
\end{quote}

\paragraph{Usage}

\paragraph{Fan-in / Fan-out}

\paragraph{Layout}

\paragraph{Files}

\paragraph{See also}


VDDIO \\
GND \\
ANA

\onecolumn
\
\end{document}
